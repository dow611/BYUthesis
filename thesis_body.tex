\chapter{Introduction}
\lipsum[3]
This idea is discussed in more detail in~\cite{guy06best}.
\section{First Idea}
\lipsum[1]
See~\cite{abramovitch90lyapunov} for more information. This idea is illustrated in~Figure~\ref{fig:triangle}.
\begin{figure}
  \centering
  \setlength{\unitlength}{0.8cm}
  \begin{picture}(6,5)
    \thicklines
    \put(1,0.5){\line(2,1){3}}
    \put(4,2){\line(-2,1){2}}
    \put(2,3){\line(-2,-5){1}}
    \put(0.7,0.3){$A$}
    \put(4.05,1.9){$B$}
    \put(1.7,2.95){$C$}
    \put(3.1,2.5){$a$}
    \put(1.3,1.7){$b$}
    \put(2.5,1.05){$c$}
    \put(0.3,4){$F=
    \sqrt{s(s-a)(s-b)(s-c)}$}
    \put(3.5,0.4){$\displaystyle
    s:=\frac{a+b+c}{2}$}
  \end{picture}
\caption{Place the caption after the figure placing it below}
\label{fig:triangle}
\end{figure}
\subsection{More Details}
\lipsum[1]
\subsection{Even More Details}
\lipsum[1]
\section{Second Idea}
\lipsum[4]
See~\cite{jazwinski70stochastic,guy06second}.

\chapter{Contributions}
\lipsum[2]
\section{First Contribution}
\lipsum[4] This idea is illustrated in Table~\ref{tab:numbers}.
\begin{table}
  \caption{Place the caption before the table placing it above}
  \centering
  \begin{tabular}{|r|l|}
    \hline
    7C0 & hexadecimal \\
    3700 & octal \\ \cline{2-2}
    11111000000 & binary \\
    \hline \hline
    1984 & decimal \\
    \hline
    \label{tab:numbers}
  \end{tabular}
\end{table}
\section{Really, really long name of second contribution that is so verbose that it spans multiple lines}
\lipsum[1]

\chapter{Really, really long chapter title that is so verbose that it spans multiple lines}
\lipsum[2]
